Note correction morel

p.11 Lemma 1 :  Il faut absolument faire une figure correspondant aux notations précédentes et à cette démonstration, où apparaissent tous  les points et vecteurs: $X$, $H(X)$, $C_0$, $w$, $\delta$, $F$.

Relire 2.2 (j’ai essayer de présenter de manière plus claire, j’ai un peu changé l’ordre et relu)
--> j'ai [shmuel] relu l'anglais mais pas le contenu mathématique, j'avoue que je ne comprends pas les calculs

Relire 2.3 et Conclusion (je n’ai rien toucher) 
Faut-il supprimer la partie 2.3 (qui est redondante avec les experiments il me semble)
--> j'ai [shmuel] supprimé la partie 2.3, j'ai fusionné son contenu avec la section correspondante des expériences. J'ai aussi supprimé la partie "b-plines" de la conclusion, parce que je pense que c'est absolument pas le sujet de l'article et ce ne sera pas utilisé (car les angles seront toujours quelconques).

p.34 : Il n'est pas possible de juger du succès de la méthode sans disposer d'une "vérité terrain" qui peut être obtenue en faisant un très gros zoom in (d'un facteur 5 ou 10) par zero-padding, suivi de l'application de l'homographie par interpolation bilinéaire, et suivi par le zoom arrière du facteur inverse, par un noyau gaussien d'écart type 0.8 n, où n est  le facteur de zoom.





La conclusion est avant les experiments... C'est étrange.
