Relire 2.2 (j’ai essayer de présenter de manière plus claire, j’ai un peu changé l’ordre et relu)
--> j'ai [shmuel] relu l'anglais mais pas le contenu mathématique, j'avoue que je ne comprends pas les calculs

p.34 : Il n'est pas possible de juger du succès de la méthode sans disposer d'une "vérité terrain" qui peut être obtenue en faisant un très gros zoom in (d'un facteur 5 ou 10) par zero-padding, suivi de l'application de l'homographie par interpolation bilinéaire, et suivi par le zoom arrière du facteur inverse, par un noyau gaussien d'écart type 0.8 n, où n est  le facteur de zoom.

premiere image spectaculaire (eg avec les briques)

La conclusion est avant les experiments... C'est étrange.[hugo]

Je suis pour supprimer la partie (homo->affine)[hugo]
